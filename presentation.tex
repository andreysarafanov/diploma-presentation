\documentclass{beamer}

\mode<presentation>
{
  \usetheme{Singapore}
  \usecolortheme{default}
}

\usepackage[english, russian]{babel}
\usepackage[utf8]{inputenc}
\usepackage{times}
\usepackage[T2A]{fontenc}
\usepackage{pscyr}
\usepackage{amsthm,amsfonts,amsmath,amssymb,amscd}
\usepackage{courier}
\usepackage{verbatim}

\newcommand{\bs}{\textbackslash}

\title {Некоторые алгоритмы, основанные на графовом представлении бесконтекстных языков}

\author {Андрей Сарафанов, группа 524}

\institute {\small{
  Факультет вычислительной математики и кибернетики\\
  МГУ им. Ломоносова
}}

\date {\footnotesize{Москва, 2015}}
\begin{document}

\begin{frame}
  \titlepage
\end{frame}

\begin{frame} {Цели работы}
\begin{itemize}
  \item Построить алгоритм построения по детерминированному бесконтекстному L-графу регулярного L-графа, эквивалентного исходному L-графу, только если исходный L-граф регулярен
  \item Построить алгоритм проверки эквивалентности детерминированного бесконтекстного L-графа и регулярного L-графа
  \item Выделить подкласс детерминированных бесконтекстных L-графов, в котором получаемый последовательным применением этих алгоритмов признак регулярности детерминированных бесконтекстных L-графов был бы критерием
\end{itemize}
\end{frame}

\begin{frame} {Результаты работы}
\begin{itemize}
  \item Построен алгоритм построения по детерминированному бесконтекстному L-графу регулярного L-графа, эквивалентного исходному  L-графу, только если тот регулярен
  \item Построен алгоритм проверки эквивалентности детерминированного бесконтекстного L-графа и регулярного L-графа
  \item На основе этих алгоритмов сформулирован признак регулярности детерминированных бесконтекстных L-графов
  \item Выделен подкласс детерминированных бесконтекстных L-графов, в котором этот признак регулярности является критерием
\end{itemize}
\end{frame}

\end{document}

