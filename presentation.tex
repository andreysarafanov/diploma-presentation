\documentclass{beamer}

\mode<presentation>
{
  \usetheme{Singapore}
  \usecolortheme{default}
}

\usepackage[english, russian]{babel}
\usepackage[utf8]{inputenc}
\usepackage{times}
\usepackage[T2A]{fontenc}
\usepackage{pscyr}
\usepackage{amsthm,amsfonts,amsmath,amssymb,amscd}
\usepackage{courier}
\usepackage{verbatim}
\addtobeamertemplate{navigation symbols}{}{%
    \usebeamerfont{footline}%
    \usebeamercolor[fg]{footline}%
    \hspace{1em}%
    \large{\insertframenumber/\inserttotalframenumber}
}
\newcommand{\bs}{\textbackslash}

\title []{Некоторые алгоритмы, основанные на графовом представлении бесконтекстных языков}

\author {Сарафанов Андрей \\ \small{группа 524\\ \vspace{1cm}Научный руководитель:\\
  к.ф.-м.н. Вылиток Алексей Александрович\\}}

%\institute {\small{
%%  Факультет вычислительной математики и кибернетики\\
%  МГУ им. Ломоносова
%}}

\date {\footnotesize{Москва, 2015}}
\begin{document}

\begin{frame}
  \titlepage
\end{frame}

\begin{frame} {Формальные языки}
Пусть задан алфавит $\Sigma$ - конечное непустое множество символов.\\
\textit{Формальный язык} - множество $L, L\subseteq \Sigma^*$.
\vspace{5mm} %5mm vertical space
\begin{columns}
  \column[T]{.55\textwidth}
    Иерархия формальных языков:
    \begin{enumerate}
      \setcounter{enumi}{-1}
      \item Рекурсивно-перечислимые
      \item Контекстно-зависимые
      \item Бесконтекстные
        \begin{itemize}
          \item Детерминированные
          \item Недетерминированные
        \end{itemize}
      \item Регулярные
    \end{enumerate}
  \column[T]{.45\textwidth}
    Способы описания формальных языков:
    \begin{itemize}
      \item Грамматики
      \item Автоматы
      \item Морфические представления
      \item Графы
    \end{itemize}
\end{columns}
\end{frame}

\begin{frame}{Представление языков в форме L-графов}
L-графы --- способ описания бесконтекстных и регулярных языков. Основаны на графовых описаниях, введенных в работых Станевичене Л.И.
\begin{columns}
  \column[T]{.55\textwidth}
    L-граф для языка $a^nbc^{n-1}$
    \begin{tikzpicture}[->,>=stealth',shorten >=1pt,auto,node distance=1cm, scale=0.8,
      thick,main node/.style={circle,fill=blue!20,draw,font=\sffamily\Large\bfseries},invis node/.style={draw=none}]
      \node[initial,state] (1) {1};
      \node[state,accepting] (2) [right= of 1] {2};

      \path[every node/.style={font=\sffamily\small}]
        (1) edge node {b|)} (2)
            edge [loop above] node {a|(} (1)
        (2) edge [loop above] node {c|)} (2);
    \end{tikzpicture}
  \column[T]{.45\textwidth}
    Примеры путей:\\
    \LARGE{$1\frac{a}{(}1\frac{a}{(}1\frac{b}{)}2\frac{c}{)}2$}
    \vspace{0.8cm}
    \LARGE{$1\frac{a}{(}1\frac{a}{(}1\frac{a}{(}1\frac{b}{)}2\frac{c}{)}2$}\\
\end{columns}
У каждого пути есть скобочная пометка $\mu(T)$ и буквенная пометка $\omega(T)$. Если скобочная пометка пути $T$ - правильная скобочная последовательность, назовём $T$ маршрутом. Тогда язык L-графа G: $L(G) = \{\omega(T)|T-$маршрут из начальной вершины G в конечную$\}$.
\end{frame}

\begin{frame} {Классификация L-графов}
\begin{itemize}
  \item Бесконтекстные L-графы\\
    Содержат скобочные пометки, эквивалентны магазинным автоматам. Делятся на:
    \begin{itemize}
      \item Детерминированные бесконтекстные L-графы
      \item Недетерминированные бесконтекстные L-графы
    \end{itemize}
  \item Регулярные L-графы\\
  Не содержат скобочных пометок, представляют из себя графы конечных автоматов. Делятся на\\
  \begin{itemize}
      \item Детерминированные регулярные L-графы
      \item Недетерминированные регулярные L-графы
    \end{itemize}
\end{itemize}
\end{frame}

\begin{frame} {Проблема регулярности магазинных автоматов}
Пусть $M$ - некий МА. Существует ли конечный автомат $A \text{ такой, что } L(A)=L(M)$?
Эта задача неразрешима.\\
\vspace{0.5cm}
Пусть теперь $M$ - некий ДМА c $q$ состояниями и $t$ магазинными символами.
  \begin{itemize}
    \item Разрешимость этой суженной задачи показана в работе Stearns, 1967. Верхняя оценка числа состояний эквивалентного конечного автомата составила {\LARGE$t^{q^{q^q}}$\par}
    \item В работе Valiant, 1975 эта оценка улучшена до {\LARGE$qt^{2qt(q^2q!)^q}$\par}
  \end{itemize}
\end{frame}

\begin{frame} {Цели работы}
\begin{itemize}
  \item Построить алгоритм построения по детерминированному бесконтекстному L-графу регулярного L-графа, эквивалентного исходному L-графу, только если исходный L-граф регулярен
  \item Построить алгоритм проверки эквивалентности детерминированного бесконтекстного L-графа и регулярного L-графа
  \item Выделить подкласс детерминированных бесконтекстных L-графов, в котором получаемый последовательным применением этих алгоритмов признак регулярности детерминированных бесконтекстных L-графов был бы критерием
\end{itemize}
\end{frame}

\begin{frame} {Результаты работы}
\begin{itemize}
  \item Построен алгоритм построения по детерминированному бесконтекстному L-графу регулярного L-графа, эквивалентного исходному  L-графу, только если тот регулярен
  \item Построен алгоритм проверки эквивалентности детерминированного бесконтекстного L-графа и регулярного L-графа
  \item На основе этих алгоритмов сформулирован признак регулярности детерминированных бесконтекстных L-графов
  \item Выделен подкласс детерминированных бесконтекстных L-графов, в котором этот признак регулярности является критерием
\end{itemize}
\end{frame}

\end{document}

