\documentclass{beamer}

\mode<presentation>
{
  \usetheme{Singapore}
  \usecolortheme{default}
}

\usepackage[english, russian]{babel}
\usepackage[utf8]{inputenc}
\usepackage{times}
\usepackage[T2A]{fontenc}
\usepackage{pscyr}
\usepackage{amsthm,amsfonts,amsmath,amssymb,amscd}
\usepackage{courier}
\usepackage{verbatim}
\addtobeamertemplate{navigation symbols}{}{%
    \usebeamerfont{footline}%
    \usebeamercolor[fg]{footline}%
    \hspace{1em}%
    \large{\insertframenumber/\inserttotalframenumber}
}
\newcommand{\bs}{\textbackslash}

\title []{Некоторые алгоритмы, основанные на графовом представлении бесконтекстных языков}

\author {Сарафанов Андрей \\ \small{группа 524\\ \vspace{1cm}Научный руководитель:\\
  к.ф.-м.н. Вылиток Алексей Александрович\\}}

%\institute {\small{
%%  Факультет вычислительной математики и кибернетики\\
%  МГУ им. Ломоносова
%}}

\date {\footnotesize{Москва, 2015}}
\begin{document}

\begin{frame}
  \titlepage
\end{frame}

\begin{frame} {Формальные языки}
Пусть задан алфавит $\Sigma$ - конечное непустое множество символов.\\
\textit{Формальный язык} - множество $L, L\subseteq \Sigma^*$.
\vspace{5mm} %5mm vertical space
\begin{columns}
  \column[T]{.55\textwidth}
    Иерархия формальных языков:
    \begin{enumerate}
      \setcounter{enumi}{-1}
      \item Рекурсивно-перечислимые
      \item Контекстно-зависимые
      \item Бесконтекстные
        \begin{itemize}
          \item Детерминированные
          \item Недетерминированные
        \end{itemize}
      \item Регулярные
    \end{enumerate}
  \column[T]{.45\textwidth}
    Способы описания формальных языков:
    \begin{itemize}
      \item Грамматики
      \item Автоматы
      \item Морфические представления
      \item Графы
    \end{itemize}
\end{columns}
\end{frame}

\begin{frame} {Проблема регулярности магазинных автоматов}
Пусть $M$ - некий МА. Существует ли конечный автомат $A \text{ такой, что } L(A)=L(M)$?
Эта задача неразрешима.\\
\vspace{0.5cm}
Пусть теперь $M$ - некий ДМА c $q$ состояниями и $t$ магазинными символами.
  \begin{itemize}
    \item Разрешимость этой суженной задачи показана в работе Stearns, 1967. Верхняя оценка числа состояний эквивалентного конечного автомата составила {\LARGE$t^{q^{q^q}}$\par}
    \item В работе Valiant, 1975 эта оценка улучшена до {\LARGE$qt^{2qt(q^2q!)^q}$\par}
  \end{itemize}
\end{frame}

\begin{frame}{Представление языков в форме L-графов}
L-графы --- способ описания бесконтекстных и регулярных языков. Основаны на графовых описаниях, введенных в работах Л.И. Станевичене
\begin{columns}
  \column[T]{.55\textwidth}
    L-граф для языка $a^nbc^{n-1}$
    \begin{tikzpicture}[->,>=stealth',shorten >=1pt,auto,node distance=1cm, initial text={}, scale=0.8,
      thick,main node/.style={circle,fill=blue!20,draw,font=\sffamily\Large\bfseries},invis node/.style={draw=none}]
      \node[initial,state] (1) {1};
      \node[state,accepting] (2) [right= of 1] {2};

      \path[every node/.style={font=\sffamily\small}]
        (1) edge node {b|)} (2)
            edge [loop above] node {a|(} (1)
        (2) edge [loop above] node {c|)} (2);
    \end{tikzpicture}
  \column[T]{.45\textwidth}
    Примеры путей:\\
    \LARGE{$1\frac{a}{(}1\frac{a}{(}1\frac{b}{)}2\frac{c}{)}2$}
    \vspace{0.8cm}
    \LARGE{$1\frac{a}{(}1\frac{a}{(}1\frac{a}{(}1\frac{b}{)}2\frac{c}{)}2$}\\
\end{columns}
У каждого пути есть скобочная пометка $\mu(T)$ и буквенная пометка $\omega(T)$. Если скобочная пометка пути $T$ - правильная скобочная последовательность, назовём $T$ маршрутом. Тогда язык L-графа G: $L(G) = \{\omega(T)|T-$маршрут из начальной вершины G в конечную$\}$.
\end{frame}

\begin{frame} {Классификация L-графов}
\begin{itemize}
  \item Бесконтекстные L-графы\\
    Содержат скобочные пометки, эквивалентны магазинным автоматам. Делятся на:
    \begin{itemize}
      \item Детерминированные бесконтекстные L-графы
      \item Недетерминированные бесконтекстные L-графы
    \end{itemize}
  \item Регулярные L-графы\\
  Не содержат скобочных пометок, представляют из себя графы конечных автоматов. Делятся на\\
  \begin{itemize}
      \item Детерминированные регулярные L-графы
      \item Недетерминированные регулярные L-графы
    \end{itemize}
\end{itemize}
\end{frame}

\begin{frame} {Цели работы}
\begin{itemize}
  \item Построить алгоритм трансформации детерминированного бесконтекстного L-графа в регулярный L-граф, эквивалентный исходному L-графу, только если исходный L-граф регулярен
  \item Построить алгоритм проверки эквивалентности детерминированного бесконтекстного L-графа и регулярного L-графа
  \item Выделить подкласс детерминированных бесконтекстных L-графов, в котором получаемый последовательным применением этих алгоритмов признак регулярности детерминированных бесконтекстных L-графов был бы критерием
\end{itemize}
\end{frame}

\begin{frame} {Нейтральные циклы и гнёзда в L-графе}
\begin{itemize}
  \item \textit{Нейтральным циклом} будет называть путь $T$, такой что $\mu(T) = \Lambda, beg(T) = end(T)$\\
  \item Пусть $T$ --- некоторый маршрут в L-графе, $T=T_1T_2T_3, \mu(T_2) = \Lambda, \mu(T_1T_2T_3) = \Lambda$, $T_1$ и $T_2$ - не нейтральные циклы. Тогда $(T_1,T_2,T_3)$ назовём \textit{гнездом} с парными циклами $T_1$ и $T_3$.
\end{itemize}
\end{frame}

\begin{frame}{Ядро L-графа}
В работе Станевичене, 2000 представлено понятие \textit{ядра} D-графа, которое можно перенести на L-графы.\\
\vspace{0.3cm}
(1,1)-ядром L-графа G ($Core(G,1,1)$) назовём множество всех его путей $T$, таких что
\begin{itemize}
  \item внутри $T$ нет участков вида $T_1T_2$, где $T_1 \text{ и } T_2$ --- нейтральные циклы
  \item внутри $T$ нет участков вида $T_1T_2T_3T_4T_5$, таких что $(T_2,T_3,T_4)$ - гнездо с парными циклами $T_2,T_4$ и $(T_1,T_2T_3T_4,T_5)$ - гнездо с парными циклами $T_1,T_5$
\end{itemize}

В этой работе показано, что (1,1)-ядро является "структурной основой"  L-графов, все маршруты L-графа можно получить, добавляя проходы по нейтральным и парным циклам к маршрутам из (1,1)-ядра.
\end{frame}

\begin{frame}{Трансформация L-графа в граф-кандидат}{Граф памятей}
Пусть $T$ - маршрут L-графа G, $T=T_1T_2, end(T_1) = v,\allowbreak \mu(T_1) = \gamma$, тогда пару $(v,\gamma)$ назовём \textit{памятью} графа G.\\
Как $CoreMem(G,w,d)$ обозначим множество памятей, входящих в маршруты из (w,d)-ядра графа G.
Построим граф конечного автомата, в котором вершины - элементы $CoreMem(G,w,d)$, а дуги построены на основе дуг из G.
\begin{columns}
  \column[T]{.55\textwidth}
    L-граф G, $L(G) = a^nbc^n$
    \begin{tikzpicture}[->,>=stealth',shorten >=1pt,auto,node distance=1cm, initial text={}, scale=0.8,
      thick]
      %\tikzset[initial text={}]
      \node[initial,state] (1) {1};
      \node[state,accepting] (2) [right= of 1] {2};

      \path[every node/.style={font=\sffamily\small}]
        (1) edge node {b} (2)
            edge [loop above] node {a|(} (1)
        (2) edge [loop above] node {c|)} (2);
    \end{tikzpicture}
  \column[T]{.45\textwidth}
    $MemGraph(G,1,1)$
    \begin{tikzpicture}[->,>=stealth',shorten >=1pt,auto,node distance=1cm, initial text={}, scale=0.8,
      thick,limited size/.style={minimum size=2.8em}]
      \node[initial,state,limited size] (1) {$1,\epsilon$};
      \node[state,accepting,limited size] (2) [right= of 1] {$2,\epsilon$};
      \node[state,limited size] (3) [below=0.7cm of 1] {$1,($};
      \node[state,limited size] (4) [right= of 3] {$2,(($};

      \path[every node/.style={font=\sffamily\small}]
        (1) edge node {b} (2)
        (1) edge node {a} (3)
        (3) edge node {b} (4)
        (4) edge node {c} (2);
    \end{tikzpicture}
  \end{columns}
\end{frame}

\begin{frame} {Трансформация L-графа в граф-кандидат}{Объединение графов памятей}
Пусть $ToRemove(G) =\allowbreak CoreMem(G,2,2)\backslash \allowbreak CoreMem(G,1,1)$. Будем исключать из $MemGraph(G,2,2)$ вершины из $ToRemove(G)$, направляя инцедентные им дуги в соответствующие вершины из $CoreMem(1,1)$. В результате получится граф-кандидат $Cand(G)$.
  \only<1>{
    \begin{columns}
      \column[T]{.45\textwidth}
        $MemGraph(G,1,1)$
        \begin{tikzpicture}[->,>=stealth',shorten >=1pt,auto,node distance=1cm, initial text={}, scale=0.8,
          thick,limited size/.style={minimum size=2.8em}]
          \node[initial,state,limited size] (1) {$1,\epsilon$};
          \node[state,accepting,limited size] (2) [below= of 1] {$2,\epsilon$};
          \node[state,limited size, draw=blue] (3) [right= of 1] {$1,($};
          \node[state,limited size, draw=orange] (4) [below= of 3] {$2,(($};

          \path[every node/.style={font=\sffamily\small}]
            (1) edge node {b} (2)
            (1) edge node {a} (3)
            (3) edge node {b} (4)
            (4) edge node {c} (2);
        \end{tikzpicture}
      \column[T]{.55\textwidth}
        $MemGraph(G,2,2)$
        \begin{tikzpicture}[->,>=stealth',shorten >=1pt,auto,node distance=1cm, initial text={}, scale=0.8,
          thick,limited size/.style={minimum size=3.1em}]
          \node[initial,state,limited size] (1) {$1,\epsilon$};
          \node[state,accepting,limited size] (2) [below= of 1] {$2,\epsilon$};
          \node[state,limited size] (3) [right=0.7cm of 1] {$1,($};
          \node[state,limited size] (4) [below= of 3] {$2,(($};
          \node[state,limited size, draw=blue] (5) [right=0.7cm of 3] {$1,(($};
          \node[state,limited size, draw=orange] (6) [below= of 5] {$2,((($};

          \path[every node/.style={font=\sffamily\small}]
            (1) edge node {b} (2)
            (1) edge node {a} (3)
            (3) edge node {b} (4)
            (3) edge node {a} (5)
            (4) edge node {c} (2)
            (5) edge node {b} (6)
            (6) edge node {c} (4);
        \end{tikzpicture}
    \end{columns}
  }
  \only<2>{
    \begin{columns}
      \column[T]{.2\textwidth}
      \column[T]{.6\textwidth}
        $Cand(G)$
        \begin{tikzpicture}[->,>=stealth',shorten >=1pt,auto,node distance=1cm, initial text={}, scale=0.8,
          thick,limited size/.style={minimum size=2.8em}]
          \node[initial,state,limited size] (1) {$1,\epsilon$};
          \node[state,accepting,limited size] (2) [below= of 1] {$2,\epsilon$};
          \node[state,limited size, draw=blue] (3) [right= of 1] {$1,($};
          \node[state,limited size, draw=orange] (4) [below= of 3] {$2,(($};

          \path[every node/.style={font=\sffamily\small}]
            (1) edge node {b} (2)
            (1) edge node {a} (3)
            (3) edge node {b} (4)
            (3) edge [loop right, draw=blue] node {a} (3)
            (4) edge node {c} (2)
            (4) edge [loop right, draw=orange] node {c} (4);
        \end{tikzpicture}
        \column[T]{.2\textwidth}
    \end{columns}
  }
\end{frame}

\begin{frame} {Трансформация L-графа в граф-кандидат}{Свойства графа-кандидата}
\begin{itemize}
  \item $L(G) \subseteq L(Cand(G))$
  \item $L(G) = L(Cand(G)) \implies G\text{ задаёт регулярный язык }$
  \item Если для любого гнезда $(T_1,T_2,T_3)$, входящего в какой-либо маршрут из $Core(G,1,1)$ либо $\omega(T_1) = \Lambda$, либо $\omega(T_3) = \Lambda$, то $$L(G) = L(Cand(G)) \Longleftrightarrow G\text{ задаёт регулярный язык }$$
\end{itemize}
\end{frame}

\begin{frame} {Проверка эквивалентности детерминированных регулярного и бесконтекстного L-графов}
Пусть
\begin{itemize}
  \item $G_{CF}$ - детерминированный бесконтекстный L-граф
  \item $G_R$ - детерминированный регулярный L-граф
\end{itemize}
{\color{red} Доделать}
\end{frame}

\begin{frame} {Результаты работы}
\begin{itemize}
  \item Построен алгоритм трансформации детерминированного бесконтекстного L-графа в регулярный L-граф, эквивалентный исходному L-графу, только если исходный L-граф регулярен
  \item Построен алгоритм проверки эквивалентности детерминированного бесконтекстного L-графа и регулярного L-графа
  \item На основе этих алгоритмов сформулирован признак регулярности детерминированных бесконтекстных L-графов
  \item Выделен подкласс детерминированных бесконтекстных L-графов, в котором этот признак регулярности является критерием
\end{itemize}
\end{frame}

\end{document}

